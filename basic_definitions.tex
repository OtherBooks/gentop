%!TEX root = rootfile.tex 
% chktex-file 3
% chktex-file 8
% chktex-file 12
% chktex-file 24
% chktex-file 42

Our starting place is the study of the \emph{subspaces} of Euclidean space.

\begin{ntn}
	As usual, $\RR$ will denote the ordered field of real numbers.
	For any $ n\geq 0$, we write $\RR^n$ for the vector space of $n$-tuples $x = (x_1,\dots,x_n)$ with $x_i \in \RR$.
	
	For our purposes, one of the most important features $\RR^n$ has to offer is its \emph{metric};
	this is the map
	\[
		d \colon \RR^n \times \RR^n \to \RR \comma
	\]
	called the \emph{distance}, given by the formula
	\[
		d(x,y) \coloneq \|x - y\| = \sqrt{\sum_{i=1}^n (x_i-y_i)^2} \period
	\]
	This map enjoys four key properties:
	\begin{enumerate}
		\item (Nonnegativity.) For every pair of points $x,y \in \RR^n$, one has $d(x,y) \geq 0$.
		\item (Identity.) For every pair of points $x,y \in \RR^n$, one has $d(x,y)=0$ if and only if $x=y$.
		\item (Symmetry.) For every pair of points $x,y \in \RR^n$, one has $d(x,y) = d(y,x)$.
		\item (Triangle inequality.) For every triple of points $x,y,z \in \RR^n$, one has
			\[
				d(x,y) + d(y,z) \geq d(x, z) \period
			\]
	\end{enumerate}

	If $ r \geq 0$ and $ x \in \RR^n $, then we define the \emph{ball of radius $ r $ centered at $x$}:
		\[
			B^n(x,r) \coloneq \{ y \in \RR^n : d(x,y) <r \} \period
		\]
\end{ntn}

In topology, we are not interested in the metric $d$ itself but in what it can tell us about the nearness of points and subsets:

\begin{dfn}
	Let $ S \subseteq \RR^n $ be a subset.
	A point $ x \in \RR^n $ will be said to be \defn{close to} $S$ if and only if, for every $ \varepsilon > 0 $, there exists an element $ s \in S $ such that $ d(x,s) < \varepsilon $.

	We may express this notion with the aid of an auxiliary quantity.
	Define the \defn{distance from $x$ to $S$} as%
	\sidenote{If $S = \varnothing$, then this infimum doesn't exist as a real number.
	So we'll declare formally that $d(x,\varnothing) = +\infty$.}
	\[
		d(x,S) \coloneq \inf \{ d(x,s) : s \in S \} \period
	\]
	Then $x$ is \defn{close to} $S$ if and only if $d(x,S) =0$.
	
	Let $ X \subseteq \RR^n$ be a subset such that $ S \subseteq X $.
	We shall write $ \tau_X(S) \subseteq X $ for the set of $ x \in X $ that are close to $ S $:
	\[
		\tau_X(S) \coloneq \{ x \in X : (\forall \varepsilon >0)(\exists s \in S)(d(x,s)<\varepsilon) \} = \{x \in X : d(x,S) = 0\}
	\]
	This is called the \defn{closure of $S$ in $X$}.
	In other words,
	\[
		\tau_X(S) = X \cap \bigcap_{\varepsilon>0}\bigcup_{s\in S} B^n(s,\varepsilon) \period
	\]

	We will refer to $X $ along with the map%
	\sidenote{The set $\PP(X)$ is the \emph{powerset} of $X$;
	it is the set of subsets of $X$.}
	$\tau_X \colon \PP(X) \to \PP(X) $ as the \emph{subspace} $X \subseteq \RR^n$.
\end{dfn}

Every time you see a new definition in mathematics, you should begin immediately to think about as many examples as you can.
You should also have a couple of \defn{counterexamples} at hand, preferably at least one for each condition in a definition.
Then, when you encounter a theorem with a proof, you can take your example and \enquote{run} the proof on your example, and you can run it on your counterexample to see where the conditions are really being used.
This is one of the most important ways to understand how theorems really work.

In that spirit, let's dig into some examples.

\begin{exm}
	\begin{itemize}
		\item If $ X = \varnothing $, then this definition becomes a little boring:
			the only subset of $\varnothing$ is $\varnothing$ itself, and $\tau_{\varnothing}(\varnothing) = \varnothing$.
		\item If $ X = \RR$ itself, then let's see what happens for various choices of $S$:
			\begin{itemize}
				\item Let's consider various unit intervals:
					\[
						\tau_{\RR}(\left]0,1\right[) = \tau_{\RR}([0,1]) = \tau_{\RR}(\left]0,1\right]) = [0,1]
					\]
				\item If $\QQ \subset \RR$ denotes the set of rational numbers, then $\tau_{\RR}(\QQ) = \RR $.
				\item At the same time, $\tau_{\RR}(\RR \smallsetminus \QQ) = \RR$.
				\item Confirm that  $\tau_{\RR}(\varnothing)=\varnothing$.
			\end{itemize}
		\item If $ X \subseteq \RR^n$ is a subspace, then we have
			\[
				\tau_X(S) = \tau_{\RR^n}(S) \cap X \period
			\]
			This is an important point: the closure of $S$ is done \emph{relative to the subspace $X$}.
			Let's see how the closure of $S = \left]0,1\right[$ depends on which subspace $X \subseteq \RR $ in which we are closing:
			\begin{itemize}
				\item If $X=\RR$, then $\tau_{\RR}(\left]0,1\right[) = [0,1]$.
				\item If $X=\left]0,1\right[$, then $\tau_{\left]0,1\right[}(\left]0,1\right[) = \left]0,1\right[$.
				\item If $X=\left]0,1\right]$, then $\tau_{\left]0,1\right]}(\left]0,1\right[) = \left]0,1\right]$.
				\item If $X=\left]0,+\infty\right[$, then $\tau_{\left]0,+\infty\right[}(\left]0,1\right[) = \left]0,1\right]$.
			\end{itemize}
	\end{itemize}
\end{exm}

\begin{exm}
	Let $X \subseteq \RR$ be a subspace.
	Let $S \subseteq X$ be a nonempty subset that is bounded above.
	The completeness property of $\RR$ then ensures that $S$ has a supremum $\sup S$.
	If $\sup S \in X$, then $\sup S$ is close to $S$ in $X$;
	that is,
	\[
		\sup S \in \tau_X(S) \period
	\]

	Let's prove this; let $x\coloneq \sup S$.
	Let $\varepsilon>0$.
	If $ \left]x-\varepsilon, x+\varepsilon\right[ \cap S = \varnothing$, then $x-\varepsilon$ is also an upper bound for $S$, contradicting the assumption that $x$ is the \emph{least} upper bound for $S$.
	Consequently, there exists an element $ s \in S$ such that $d(x,s) < \varepsilon$.

	The same argument shows that if $S \subseteq X $ is nonempty and bounded below, then if $\inf S \in X$, then
	\[
		\inf S \in \tau_X(S) \period
	\]
\end{exm}

\newthought{Here are} the basic facts about the closure operator $\tau_X \colon \PP(X) \to \PP(X)$.
As we shall see, these facts are in many ways even more important than the definition of $\tau_X$ itself.
\begin{prp}%
\label{prp:key_properties_of_tau}
	Let $X \subseteq \RR^n $ be a subspace.
	\begin{enumerate}
		\item For every subset $S \subseteq X $, every point of $S$ is close to $S$.
			That is, $S \subseteq \tau_X(S)$.
		\item No points of $X$ are close to $\varnothing$.
			That is, $\tau_X(\varnothing) = \varnothing$.
		\item If $S_1,\dots,S_n \subseteq X$ are a finite collection of subsets, then a point $x \in X$ is close to $ S_1 \cup \dots \cup S_n $ if and only if, for some $i$, the point $x$ is close to $S_i$.
			That is,
			\[
				\tau_X\left(S_1 \cup \dots \cup S_n\right) = \tau_X(S_1) \cup \dots \cup \tau_X(S_n) \period
			\]
		\item If $S \subseteq X$, then if $x \in X$ is a point that is close to $\tau_X(S)$, then $x$ is close to $S$ itself.
			That is, $\tau_X(\tau_X(S)) = \tau_X(S)$.
	\end{enumerate}
\end{prp}

\begin{proof}
	Let us prove these in order.
	\begin{enumerate}
		\item If $ x \in S$, then for every $\varepsilon>0$, we certainly have $d(x,x) = 0 < \varepsilon$.
		\item For every $\varepsilon>0$, the sentence%
			\sidenote{Sentences of the form $(\exists s \in \varnothing)(\phi(s))$ are \emph{always} false, no matter what $\phi(s)$ says!}
			$(\exists s \in \varnothing)(d(x,s)<\varepsilon)$ is false.
		\item If $ x $ is close to $ S_1 \cup \dots \cup S_n $, then for every natural number%
			\sidenote{In this text, $\NN$ will denote the natural numbers $\{0,1,\dots\}$, and $\NN^{\ast}$ will denote the natural numbers $\{1,2,\dots\}$.}
			$ m \in \NN^{\ast} $, we may select $s_m \in S_1 \cup \dots \cup S_n $ such that $ d(x, s_m) < 1/m$.
			By the pigeonhole principle, there exists an $i$ such that the set $\{ m \in \NN^{\ast} : s_m \in S_i \}$ is infinite.
			Consequently, for any $\varepsilon>0$, there exists a natural number $m\geq 1$ such that $ s_m \in S_i$ and $1/m < \varepsilon$.
			In particular
			\[
				d(x,s_m) < 1/m < \varepsilon \period
			\]
			Thus $ x $ is close to $S_i$.

			Conversely, if $ x $ is close to some $S_i$, then for every $\varepsilon >0$, there exists $ s \in S_i \subseteq S_1 \cup \dots \cup S_n $ such that $ d(x,s) <\varepsilon $,
			so $ x $ is close to $S_1 \cup \dots \cup S_n$.
		\item If $ x $ is close to $ \tau_X(S)$, then for any $\varepsilon >0$, we may select $ t\in \tau_X(S)$ such that $d(x,t) < \varepsilon/2 $.
			Since $t$ is close to $S$, we may select $s \in S$ such that $d(t, s) < \varepsilon/2 $.
			Now by the triangle inequality,
			\[
				d(x, s) \leq d(x, t) + d(t, s) < \varepsilon \period
			\]
			Thus $ x$ is close to $S$.
	\end{enumerate}
\end{proof}

\begin{cor}%
\label{cor:tau_preserves_inclusion}
	If $X \subseteq \RR^n$ is a subspace, then for any subsets $S, T \in \PP(X)$, if $S \subseteq T$, then $\tau_X(S) \subseteq \tau_X(T)$.
\end{cor}

\begin{proof}
	We may write $ T = S \cup (T \smallsetminus S) $.
	Then by the third point of the proposition above,
	\[
		\tau_X(T) = \tau_X(S) \cup \tau_X(T \smallsetminus S) \comma
	\]
	and so $ \tau_X(S) \subseteq \tau_X(T)$.
\end{proof}

If $X \subseteq \RR^n$ is a subspace, then there are two different kinds of subset that play a big role in topology:
\begin{dfn}
	A subset $Z \subseteq X$ is \emph{closed in $X$} if and only if
	\[
		\tau_X(Z) = Z \period
	\]
	In other words, $Z$ is closed if and only if every point of $X$ that is close to $Z$ is contained in $Z$.

	A subset $U \subseteq X$ is \defn{open in $X$} if and only if $X \smallsetminus U $ is closed.
	In other words, $U$ is open if and only if no point of $U$ is close to the complement of $U$.
\end{dfn}

\begin{exm}
	\begin{itemize}
		\item For any subspace $X \subseteq \RR^n$, the subset $X \subseteq X$ is both open and closed.%
			\sidenote{
				This is sometimes a source of confusion -- or at least irritation.
				One may expect that \enquote{open} and \enquote{closed} are opposites, but they are not:
				there are subsets that are neither, and there are subsets that are both.
				Instead, they are \emph{complementary} notions.
			}
		\item For any subspace $X \subseteq \RR^n$, the subset $\varnothing \subseteq X $ is both open and closed.
		\item In $\RR$ itself, the subset $[0,1] \subset \RR$ is closed but not open.
		\item In $\RR$, the subset $\left]0,1\right[ \subset \RR$ is open but not closed.
		\item In $\RR$, the subsets $\left[0,1\right[$ and $\left]0,1\right]$ are neither open nor closed.
		\item In the subspace $\left]0, +\infty\right[ \subseteq \RR$, the subset $\left]0,1\right] \subseteq \left]0, +\infty\right[$ is closed but not open.
		\item In the subspace $\left[0, +\infty\right[ \subseteq \RR$, the subset $\left[0,1\right[ \subseteq \left[0, +\infty\right[$ is open but not closed.
	\end{itemize}
\end{exm}

\begin{prp}%
\label{prp:tau_is_smallest_closed}
	For any subspace $X \subseteq \RR^n$ and any subset $ S \subseteq X $, the closure $\tau_X(S)$ is the smallest closed subset%
	\sidenote{That is, $\tau_X(S) \subseteq X$ is a closed subset that contains $S$, and for any closed subset $Z \subseteq X$ that contains $S$, one has $\tau_X(S) \subseteq Z$.}
of $X$ that contains $S$.
\end{prp}

\begin{proof}
	Let $Z \subseteq X $ be a closed subset that contains $S$.
	Then
	\[
		\tau_X(S) \subseteq \tau_X(Z) = Z \period
	\]
	Hence $\tau_X(S)$ is contained in any closed subset of $X$ that contains $S$.
	It remains to note that $\tau_X(S)$ is itself a closed subset of $X$ that contains $S$.
	By the proposition above,
	\[
		\tau_X(\tau_X(S)) = \tau_X(S) \comma 
	\]
	so $\tau_X(S)$ is closed.
\end{proof}

\begin{prp}%
\label{prp:balls_form_a_basis}
	Let $ X \subseteq \RR^n $ be a subspace.
	Then $ U \subseteq X $ is open if and only if, for every $ x \in U$, there exists an $\varepsilon >0$ such that
	\[
		B^n(x, \varepsilon) \cap X \subseteq U \period
	\]
\end{prp}

\begin{proof}
	Suppose that $U \subseteq X $ is open, and let $ x \in U $ be a point.
	Then by definition, $x$ is not close to $ X \smallsetminus U $.
	Hence there exists $\varepsilon>0$ such that for any point $ y \in X \smallsetminus U $, one has $d(x,y) \geq \varepsilon$.
	That is, $U$ contains $B^n(x,\varepsilon) \cap X $.

	Now assume that for every $ x \in U$, there exists an $\varepsilon >0$ such that
	\[
		B^n(x, \varepsilon) \cap X \subseteq U \period
	\]
	We aim to show that $ X \smallsetminus U $ is closed;
	hence let $ x \in U $ be a point that is close to $ X \smallsetminus U $.
	For some $\varepsilon>0$, the intersection $B^n(x,\varepsilon) \cap X $ is contained in $U$.
	But at the same time, since $ x $ is close to $X \smallsetminus U$, the intersection $B^n(x, \varepsilon) \cap (X \smallsetminus U) \neq \varnothing$.
	This is a contradiction that shows that no point of $U$ is close to $X \smallsetminus U $;
	thus $U$ is open.
\end{proof}

\begin{prp}%
\label{prp:closeds_and_open_in_subspace_topology}
	Let $X \subseteq \RR^n$ be a subspace.
	Then a subset $Z \subseteq X $ is closed if and only if $Z = Z' \cap X $ for some closed subset $Z' \subseteq \RR^n$.
	Similarly, a subset $U \subseteq X $ is open if and only if $U = U' \cap X$ for some open subset $U' \subseteq \RR^n$.
\end{prp}

\begin{proof}
	Assume that $Z' \subseteq \RR^n$ is a closed subset.
	Now consider the intersection $Z' \cap X$.
	The closure $\tau_{\RR^n}(Z' \cap X)$ is contained in $\tau_{\RR^n}(Z') = Z'$.
	Consequently,
	\[
		\tau_X(Z'\cap X) = \tau_{\RR^n}(Z'\cap X) \cap X \subseteq Z' \cap X \comma
	\]
	so $Z' \cap X $ is closed.
	
	Conversely, assume that $Z \subseteq X$ is a closed subset.
	Then $Z' \coloneq \tau_{\RR^n}(Z)$ is closed in $\RR^n$, and
	\[
		Z = \tau_X(Z) = \tau_{\RR^n}(Z) \cap X = Z' \cap X.
	\]

	The statements about open subsets of $X$ now follow by taking complements.%
	\sidenote{For proofs of this kind, it is usually enough just to offer such a short sentence.
	In this case, we will unpack this sentence a little more.}
	Indeed, if $U' \subseteq \RR^n$ is an open subset, then let $Z' $ denote its complement.
	Since $Z'\cap X$ is closed by what we've already proved, its complement in $X$, which is $U' \cap X$ is open.
	Conversely, if $U \subseteq X$ is open, then let $Z$ denote its complement in $X$.
	By what we've already shown, there is a closed subset $Z' \subseteq \RR^n$ such that $Z = Z' \cap X$, so if $U'$ denotes the open complement of $Z'$, then $U = U'\cap X$.
\end{proof}

\begin{prp}%
\label{prp:stability_properties_of_closeds_and_opens}
	Let $ X \subseteq \RR^n$ be a subspace.
	For any family $\{Z_a\}_{a\in A}$ of closed subsets of $X$, the intersection $ \bigcap_{a\in A} Z_a $ is closed as well;
	if $A$ is \emph{finite}, then the union $\bigcup_{a\in A} Z_a $ is closed as well.
	Dually, for any family $\{U_a\}_{a\in A}$ of open subsets of $X$, the intersection $ \bigcup_{a\in A} U_a $ is open as well;
	if $A$ is \emph{finite}, then the intersection $\bigcap_{a\in A} U_a $ is open as well.
\end{prp}

\begin{proof}
	Let $W = \bigcap_{a\in A} Z_a$.
	We aim to prove that $\tau_X(W) \subseteq W$.
	Let $x \in X$ be a point that is close to $W$.
	For every $a \in A$, since $W \subseteq Z_a$, it follows that $x$ is close to $Z_a$ as well.
	Since $Z_a$ is closed, $x \in Z_a$.
	This happens for every $a \in A$, so $x \in W$.

	If $A$ is finite, then we have shown that
	\[
		\tau_X\left(\bigcup_{a \in A} Z_a\right) = \bigcup_{a\in A} \tau_X(Z_a) = \bigcup_{a \in A} Z_a \comma
	\]
	so the union is closed.

	The dual statement follows by forming complements.
\end{proof}

\begin{exm}
	Let's consider examples in $\RR$ in which the finiteness hypotheses of the previous proposition are necessary.
	\begin{itemize}
		\item Infinite unions of closed subsets need not be closed:
			\[
				\bigcup_{n\in\NN^{\ast}} \left[-1+\frac{1}{n}, 1-\frac{1}{n}\right] = \left]-1,1\right[ \comma
			\]
			or open:
			\[
				\bigcup_{n\in\NN^{\ast}} \left[0,1-\frac{1}{n}\right] = \left[0,1\right[ \period.
			\]
		\item Infinite intersections of open subsets need not be open:
			\[
				\bigcap_{n\in\NN^{\ast}} \left]-1-\frac{1}{n}, 1+\frac{1}{n}\right[ = [-1,1] \comma
			\]
			or closed:
			\[
				\bigcap_{n\in\NN^{\ast}} \left]0, 1+\frac{1}{n}\right[ = \left]0,1\right] \period
			\]
	\end{itemize}
\end{exm}

We conclude with some examples of subspaces of Euclidean space to which we will return regularly.

\begin{exm}
	\begin{itemize}
		\item For any $n \in \NN$, the \defn{$n$-sphere} $S^n$ is the subspace
			\[
				S^n \coloneq \{ x \in \RR^{n+1} : \|x\| = 1 \} \period
			\]
		\item The \defn{open unit ball} is the subspace
			\[
				B^n \coloneq B^n(0,1) = \{ x \in \RR^n : \| x \| < 1 \} \comma
			\]
			and its closure in $\RR^n$ is the \defn{closed unit ball}
			\[
				\overline{B}^n \coloneq \{ x \in \RR^n : \| x \| \leq 1 \} \period
			\]
		\item If we have a finite collection of subspaces $X_1 \subseteq \RR^{n_1}, \dots, X_k \subseteq \RR^{n_k} $ then we obtain the subspace
			\[
				X_1 \times \cdots \times X_k \coloneq \{ x = (x^1, \dots, x^k) \in \RR^{n_1} \times \cdots \times \RR^{n_k} : (\forall i)(x^i \in X_i) \} \period
			\]
		\item For any $ n\in \NN $, the \defn{$n$-torus} is the subspace
			\[
				T^n \coloneq S^1 \times \cdots \times S^1 \subseteq \RR^{2n} \period
			\]
			That is,
			\[
				T^n \coloneq \{ (x_1, y_1, \dots, x_n, y_n) \in \RR^{2n} : (\forall i)(x_i^2 +y_i^2=1) \} \period
			\]
	\end{itemize}
\end{exm}

