% !TeX program = xelatex

\documentclass[a4paper,twoside,nols,nobib]{tufte-handout}
\usepackage{classnotes}

%-------------------------------------------------------------------%
% Title page data %
%-------------------------------------------------------------------%

\title{General topology}
\author{The Problems}
\date{Autumn 2020}

%-------------------------------------------------------------------%

\begin{document}

\maketitle

%-------------------------------------------------------------------%
\section*{Quotients}
%-------------------------------------------------------------------%

\begin{ntn*}
	We're going to study an interesting way to construct some topological spaces.
	Choose an integer $n \geq 3$.
	We will start with the $n$-th roots of unity in $\CC$:
	\[
		\zeta_n^0, \zeta_n^1, \dots, \zeta_n^{n-1} \comma
	\]
	where $\zeta_n \coloneq \exp(2\pi i/n)$.
	Let $P \subset \CC$ be the \emph{convex hull} of these points:
	\[
		P \coloneq \{ a_0\zeta_n^0+\cdots+a_{n-1}\zeta_n^{n-1} \in \CC : (\forall i)((a_i \in \RR_{\geq 0}) \wedge (a_1+\cdots+a_n=1)) \} \period
	\]
	This is a (filled) regular $n$-gon inscribed in the unit circle in $\CC$.
	We will also consider the \emph{interior} $\iota P$, which, recall, is the largest open subset of $\CC$ that is contained in $P$.

	A \defn{gluing datum} $(\phi, p)$ consists of an \emph{involution}%
	\sidenote{that is, a map $\phi \colon \{0,\dots,n-1\} \to \{0,\dots,n-1\} $ such that $\phi^2=\textit{id}$.}
	$\phi$ on $\{0,\dots,n-1\}$ along with a map $a \colon \{0,\dots,n-1\} \to \{+,-\}$ such that for every $i$, one has $a(i) = a(\phi(i))$, and if $i = \phi(i)$, then $a(i) = +$.

	As a bit of notation, we'll write $\phi$ as a product of disjoint transpositions:%
	\sidenote{This is possible because $\phi$ is an involution!}
	\[
		(j_1\ j_2)(j_3\ j_4) \cdots (j_{k-1}\ j_k) \comma
	\]
	and we'll add the information of the map $a$ by decorating each transposition $(j_{m-1}\ j_m)$ with a plus or minus, according to whether $a(i_m) $ is $+$ or $-$.
	Thus if $n=5$, for example,
	\[
		(0\ 3)^+(2\ 4)^-
	\]
	represents the gluing datum in which $\phi = (0\ 3)(2\ 4) $ and $a(0)=a(1)=a(3)=+$, and $a(2)=a(4) = -$.

	We'll consider the equivalence relation $\sim$ on $P$ generated by the following requirements:
	\begin{itemize}
		\item For every $j$ such that $a(j) = +$, and for every $t \in [0,1]$, we require that
			\[
				t\zeta_n^j + (1-t)\zeta_n^{j+1} \sim t \zeta_n^{\phi(j)} + (1-t) \zeta_n^{\phi(j)+1} \period
			\]
		\item For every $j$ such that $a(j) = -$, and for every $t \in [0,1]$, we require that
			\[
				t\zeta_n^j + (1-t)\zeta_n^{j+1} \sim (1-t) \zeta_n^{\phi(j)} + t \zeta_n^{\phi(j)+1} \period
			\]
	\end{itemize}
	We can then form the quotient topological space%
	\sidenote{On the final page of this PDF, we have a drawing of the identifications made by $\sim$ when $n = 4$ and the gluing datum is $(0\ 2)^-(1\ 3)^+$.}
	\[
		\Sigma \coloneq P/{\sim} \comma
	\]
	which depends on $n$, $\phi$, and $a$.
\end{ntn*}

%-------------------------------------------------------------------%

\begin{Problem}
	\noindent
	Let $n=3$.
	Describe all the different possibilities for gluing data, and describe all the different topological spaces $\Sigma$ that result from these choices.
	You need not justify your answer.
\end{Problem}

%-------------------------------------------------------------------%

\begin{Problem}
	\noindent
	Let $n = 4$.
	In the following examples, indicate whether the topological spaces $\Sigma$ are homeomorphic to a topological space we have already seen.
	\begin{itemize}
		\item $(0\ 2)^-$
		\item $(0\ 2)^+$
		\item $(0\ 2)^- (1\ 3)^-$
		\item $(0\ 2)^- (1\ 3)^+$
		\item $(0\ 2)^+ (1\ 3)^+$
	\end{itemize}
	Again, you need not justify your answer.
\end{Problem}

%-------------------------------------------------------------------%

\begin{Problem}
	Let $n = 2m$, and consider the gluing datum given by
	\[
		(0\quad m)^- (1\quad m+1)^- \cdots (m-1\quad 2m-1)^- \period
	\]
	What does the resulting topological space $\Sigma$ look like?
	Once again, you need not justify your answer.
\end{Problem}

%-------------------------------------------------------------------%

\begin{ntn*}
	Let $n \in \NN^{\ast}$.
	Recall that $\PP_{\RR}^n$ is the set of $1$-dimensional $\RR$-linear subspaces of $\RR^{n+1}$;
	we call these subspaces \defn{(real) lines}.
	Define a map
	\[
		q \colon \RR^{n+1}\smallsetminus \{0\} \to \PP_{\RR}^n
	\]
	that carries any $\mbfx = (x_0,\dots,x_n) \in \RR^{n+1}\smallsetminus \{0\}$ to the line $[\mbfx] = [x_0,\dots,x_n] \in \PP_{\RR}^n$ spanned by $\mbfx$.
	Endow $\PP_{\RR}^n$ with the finest topology such that $q$ is continuous.

	Equivalently, $\PP_{\RR}^n$ is the quotient space $\left(\RR^{n+1}\smallsetminus \{0\}\right)/\!\!\sim$, where we declare $\mbfx \sim \mbfy$ if and only if there exists a nonzero real number $\lambda$ such that $\mbfx = \lambda \mbfy$.
	
	Let $\PP^n$ denote the set of $1$-dimensional $\CC$-linear subspaces of $\CC^{n+1}$;
	we call these subspaces \defn{complex lines}.%
	\sidenote{These subspaces have $1$ complex dimension but two real dimensions.}
	Define a map
	\[
		q \colon \CC^{n+1}\smallsetminus \{0\} \to \PP^n
	\]
	that carries any $\mbfw = (w_0,\dots,w_n) \in \CC^{n+1}\smallsetminus \{0\}$ to the line $[\mbfw] = [w_0,\dots,w_n] \in \PP^n$ spanned by $\mbfw$.
	Endow $\PP^n$ with the finest topology such that $q$ is continuous.

	Equivalently, $\PP^n$ is the quotient space $\left(\CC^{n+1}\smallsetminus \{0\}\right)/\!\!\sim$, where we declare $\mbfw \sim \mbfz$ if and only if there exists a nonzero complex number $\lambda$ such that $\mbfw = \lambda \mbfz$.
\end{ntn*}

%-------------------------------------------------------------------%

\begin{Problem}
	\noindent
	Construct a homeomorphism between $\PP^2_{\RR}$ and the quotient space $\Sigma$ formed as described above by selecting $n=4$ and gluing data
	\[
		(0\ 2)^+(1\ 3)^+ \period
	\]
\end{Problem}

%-------------------------------------------------------------------%

\begin{Problem}
	\noindent
	Prove that $\PP^1$ and $S^2$ are homeomorphic.
\end{Problem}

%-------------------------------------------------------------------%

\begin{Problem}
	\noindent
	Prove that $\PP^1_{\RR}$ and $\PP^2_{\RR}$ are not homeomorphic.
\end{Problem}

%-------------------------------------------------------------------%

\begin{Problem}
	\noindent
	Find an open subset $U \subset \PP^n$ such that $U$ is homeomorphic to $\CC^n$ and such that the complement $\PP^n \smallsetminus U$ is homeomorphic to $\PP^{n-1}$.
\end{Problem}

%-------------------------------------------------------------------%

\begin{Problem}
	\noindent
	Identify $S^{2n+1}$ with the subspace
	\[
		\left\{ \mbfw \in \CC^{n+1} : \|\mbfw\| = 1\right\} \subset \CC^{n+1} \smallsetminus \{0\} \period
	\]
	Write $h$ for the composite map $S^{2n+1} \subset \CC^{n+1} \smallsetminus \{0\} \to \PP^n$.
	Prove that $h$ is surjective, and identify the fibres $h^{-1}\{x\}$, up to homeomorphism.
\end{Problem}


%-------------------------------------------------------------------%
%-------------------------------------------------------------------%
%-------------------------------------------------------------------%
%-------------------------------------------------------------------%

\end{document}
