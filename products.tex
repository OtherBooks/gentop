%!TEX root = rootfile.tex 
% chktex-file 3
% chktex-file 8
% chktex-file 12
% chktex-file 24
% chktex-file 42

\begin{nul}
	The definition of a \defn{product of sets} is engineered precisely so that a map into the product is determined by each of its components.
	More precisely, for any indexed family of sets $\{X_a\}_{a\in A}$, the product
	\[
		\prod_{a\in A}X_a
	\]
	is a set equipped with projection maps
	\[
		\pi_b\colon\fromto{\prod_{a\in A}X_a}{X_b} \comma
	\]
	one for every $b\in A$, which together \emph{determine} the maps into $\prod_{a\in A}X_a$.
	That is, composition with the projections induce a bijection
	\[
		\prod_{a\in A}\pi_{a}\circ -\colon\fromto{\Map\left(Y,\prod_{a\in A}X_a\right)}{\prod_{a\in A}\Map(Y,X_a)} \period
\]
\end{nul}

If the sets $X_a$ are all endowed with topologies, then we would like to topologise the product $\prod_{a\in A}X_a$ so that the same thing will happen with \emph{continuous} maps.
That is, we would like to ensure that a map $f\colon\fromto{Y}{\prod_{a\in A}X_a}$ is continuous if and only if, for every $a\in A$, the composite map $\pi_a\circ f\colon\fromto{Y}{X_a}$ is continuous.
But we know just how to arrange that:

\begin{dfn}
	Let $\{(X_a,\tau_a)\}_{a\in A}$ be a family of topological spaces.
	Then the \defn{product topology} $\prod_{a\in A}\tau_a$ on $\prod_{a\in A}X_a$ is the initial topology with respect to the set $\{\pi_a\}_{\alpha\in A}$.
\end{dfn}

\begin{nul}
	The definition of the \defn{product topology} is engineered precisely so that a continuous map into the product is determined by each of its continuous components.
	More precisely, for any indexed family of topological spaces $\{X_a\}_{a\in A}$, the product
	\[
		\prod_{a\in A}X_a
	\]
	is a set equipped with continuous projection maps
	\[
		\pi_b\colon\fromto{\prod_{a\in A}X_a}{X_b} \comma
	\]
	one for every $b\in A$, which together \emph{determine} the maps into $\prod_{a\in A}X_a$.
	That is, composition with the projections induce a bijection
	\[
		\prod_{a\in A}\pi_{a}\circ -\colon\fromto{\Map\left(Y,\prod_{a\in A}X_a\right)}{\prod_{a\in A}\Map(Y,X_a)} \comma
	\]
\end{nul}

\begin{exm}
	In particular, for any set $A$ and any topological space $X$, we may contemplate the product topology on the set $\Map(A,X)$.
	We write in particular
	\[
		X^n\coloneq\Map(\{1,\dots,n\},X)\text{\quad and\quad}X^{\omega}\coloneq\Map(\NN,X).
	\]
\end{exm}

\begin{prp}
	For any $n\geq 0$, the product topology on $\RR^n$ coincides with the standard topology (from the Euclidean metric).
\end{prp}
\begin{proof}
	The product topology on $\RR^n$ is generated by sets of the form $\RR\times\cdots\times \left]a,b\right[\times\RR\times\cdots\times\RR$.
	Consequently, the sets of the form $C(x,\varepsilon) \coloneq \prod_{i=1}^n\left]x_i-\varepsilon,x_i+\varepsilon\right[$ are a base for the product topology.
	Now for any $x \in \RR^n$, and for any $\varepsilon>0$, the exists a $\delta>0$ such that
	\[
		B(x,\delta) \subseteq C(x,\varepsilon) \comma
	\]
	and
	\[
		C(x,\delta) \subseteq B(x,\varepsilon) \period
	\]
	Since the subsets $C(x, \varepsilon)$ generate the product topology, and the subsets $B(x,\varepsilon)$ generate the standard topology, the two topologies coincide.
\end{proof}

\begin{wrn}
	Consider the product topological space
	\[
		\RR^{\omega} = \Map(\NN,\RR) = \prod_{n \in \NN} \RR \period
	\]
	This topology has a base
	\[
		\left\{ \prod_{i=0}^N \left]a_i, b_i\right[ \times \prod_{i=N+1}^{\infty} \RR \right\} \period
	\]
	Consequently, if $U$ is an open subset of $\RR^{\omega}$,
	then for any point $x \in U$, there exists $N \in \NN$ such that any point of the form
	\[
		(x_0,x_1,\dots,x_N,y_{N+1},y_{N+2},\dots)
	\]
	also lies in $U$.
	Conequently, the subset
	\[
		\left]0,1\right[^{\omega} = \Map(\NN, \left]0,1\right[) \subset \RR^{\omega}
	\]
	is not open!
\end{wrn}

\begin{exm}
	If $S_1,\dots,S_n$ is a finite collection of discrete topological spaces, then the product
	\[
		\prod_{i=1}^n S_i = S_1 \times \cdots \times S_n
	\]
	is also discrete.
	Indeed, for any $i$ and any $ x_i \in S_i $, the set $S_1 \times \cdots \times S_{i-1} \times \{x_i\} \times S_{i+1} \times \cdots \times S_n $ is open, and since finite intersections of opens are open, it follows that any singleton $\{(x_1, \dots, x_n)\}$ is open as well.
\end{exm}

\begin{wrn}
	If $\{S_i\}_{i\in\NN}$ is a countable family of discrete finite sets of cardinality at least $2$, the product $S \coloneq \prod_{i \in \NN} S_i$ is not discrete:
	if $(x_0, x_1, \dots) \in S$ is a point, then the open sets are unions of sets of the form
	\[
		\{x_0\} \times \cdots \times \{x_N\} \times \prod_{i=N+1}^{\infty} S_i \comma
	\]
	so that no singleton is open.
	In fact, it is always homeomorphic to our old pal the Cantor space $C$ -- irrespective of which finite sets $S_i$ are chosen!%
	\sidenote{This is not obvious, or even easy to prove.
	I personally found this fact totally implausible the first time I read it.}
\end{wrn}

There is a \emph{relative} form of the product as well.
\begin{dfn}
	Let $U$, $V$, and $X$ be three topological spaces, and let $f\colon\fromto{U}{X}$ and $g\colon\fromto{V}{X}$ two continuous maps.
	Then the \defn{fiber product}%
	\sidenote{Please notice that this notation is a little sloppy.
		The fiber product $U \times_X V$ depends on the maps $f $ and $g$.
		If one needs to make the choice of maps explicit, one may write
		\[
			U \operatornamewithlimits{\times}_{f,X,g} V \comma
		\]
	which is a bit busy, but very clear!}
	$U\times_XV$ is the subspace
	\[
		\left\{(u,v)\in U\times V : f(u)=g(v)\right\}
	\]
	of $U\times V$. In this case, the square
	\begin{equation*}
		\begin{tikzpicture}[baseline]
			\matrix(m)[matrix of math nodes,
			row sep=4ex, column sep=4ex,
			text height=1.5ex, text depth=0.25ex]
			{U\times_XV & V \\
			U & X \\};
			\path[>=stealth,->,font=\scriptsize]
			(m-1-1) edge node[above]{$\pr_2$} (m-1-2)
			edge node[left]{$\pr_1$} (m-2-1)
			(m-1-2) edge node[right]{$g$} (m-2-2)
			(m-2-1) edge node[below]{$f$} (m-2-2);
		\end{tikzpicture}
	\end{equation*}
	is sometimes called a \defn{pullback square}.
\end{dfn}

\begin{exm}
	As a special case of this construction, if $V $ is a subspace of $X$ and if $g$ is the inclusion map $ V \inclusion X$, then the fiber product $U \times_X V $ is homeomorphic to the subspace $f^{-1}(V) \subseteq X $.
	In particular, if $V = \{x\}$ for some $ x\in X$, then the fiber product $U \times_X V$ is the fiber $f^{-1}\{x\}$.
\end{exm}

\begin{exm}
	Let $f \colon X \to Y $ be a continuous map of topological spaces.
	Then the \defn{graph} of $f$ is the fiber product
	\[
		\Gamma(f) \coloneq X \operatornamewithlimits{\times}_{f,Y,\id} Y \subseteq X \times Y \period
	\]
	The projection map $ \pr_1 \colon \Gamma(f) \to X $ is a homeomorphism.
\end{exm}



