%!TEX root = rootfile.tex 
% chktex-file 3
% chktex-file 8
% chktex-file 12
% chktex-file 24
% chktex-file 42

There are a lot of different characterisations of compactness out there.
We'll give them all,%
\sidenote{or, at least, all the ones I know about!}
and we'll prove their equivalence.%
\sidenote{This is an important principle in mathematics:
	\defn{never have only one definition of any notion}.
	Instead, when possible, have multiple equivalent characterisations of the same idea.
Then use whichever one is best adapted for whichever problem you happen to encounter.}

\begin{dfn}
	Suppose $X$ and $Y$ two topological spaces.
	Then a map $f\colon\fromto{X}{Y}$ is \defn{closed} if and only if, for any subset $S\subseteq X$ one has $f(\overline{S})\supseteq\overline{f(S)}$.
	In other words, if $y$ is close to $f(S)$, then $y=f(x)$ for some $x$ that is close to $S$.
\end{dfn}

\begin{exm}
	Any homeomorphism is closed.
\end{exm}

\begin{exm}
	Suppose $X$ a topological space, and suppose $A\subseteq X$ a subspace thereof.
	Then the inclusion map $\into{A}{X}$ is closed if and only if $A$ is closed in $X$.
\end{exm}

\begin{exm} The map $f\colon\fromto{\RR}{\RR}$ given by the formula $f(x)=|x|$ is closed and, of course, continuous.
\end{exm}

\begin{exm}
	The map $s\colon\fromto{\RR}{\RR}$ given by the formula
	\[
		s(x)\coloneq\begin{cases} -1&\text{if }x<0;\\
			0&\text{if }x=0;\\
			+1&\text{if }x>0
		\end{cases}
	\]
	is closed but not continuous.
\end{exm}

\begin{prp}
	Suppose $X$ and $Y$ two topological spaces.
	Then a map $f\colon X \to Y$ is closed if and only if, for any closed subset $Z\subseteq X$, the direct image $f(Z)$ is closed.
\end{prp}

\begin{proof}
	Suppose $f$ closed, and suppose $Z\subseteq X$ a closed subset;
	then
	\[
		f(Z)=f(\overline{Z})\supseteq\overline{f(Z)},
	\]
	but the other inclusion $f(Z)\subseteq\overline{f(Z)}$ is automatic.
	Hence $f(Z)=\overline{f(Z)}$.

	Conversely, suppose that $f$ satisfies the condition that for any closed subset $Z\subseteq X$, the direct image $f(Z)$ is closed, and suppose $S\subseteq X$ any subset.
	Then $f(\overline{S})$ is certainly closed, and of course it contains $f(S)$;
	hence it contains $\overline{f(S)}$.
\end{proof}

\begin{lem}
	If $ f\colon X \to Y $ is a closed, continuous surjection, then $f$ is a quotient map.
\end{lem}

\begin{proof}
	Let $Z \subseteq Y$ be a subset such that $f^{-1}(Z) \subseteq X$ is closed.
	Then $ Z = f(f^{-1}(Z)) \subseteq Y $ is closed, since $f$ is closed.
\end{proof}

\begin{exm}
	The projection $\pr_1\colon\fromto{\RR^2}{\RR}$ (given by $\pr_1(x,y)=x$), which is certainly continuous, is not closed.%
	\sidenote{It is, however, an \defn{open map}: the direct image of any open subset is open.}
	To see this, consider the closed subset
	\[
		Z\coloneq\{(x,y)\in\RR^2 : y=2^x\};
	\]
	then $\pr_1(Z)=\left]0,+\infty\right[$, which is not closed in $\RR$.
\end{exm}

\begin{dfn}
	A \defn{cover} of a set $X$ is a subset $\UU\subseteq\PP(X)$ such that $\cup\UU=X$.
	An \defn{open cover} of a topological space $X$ is a cover whose members are all open in $X$.
\end{dfn}

\begin{prp}
	Let $X$ and $Y$ be two topological spaces, and
	let $\{W_1,\dots,W_n\}$ be a finite cover of $X$.
	Then a map $f\colon\fromto{X}{Y}$ is closed if each restriction $f|_{W_j}\colon\fromto{W_j}{Y}$ is closed.
\end{prp}

\begin{proof}
	Assume that each restriction $f|_{W_j}$ is closed, and
	let $Z\subseteq X$ a closed subset.
	Then $Z\cap W_j$ is closed in $W_j$, and so the subset
	\[
		f(Z\cap W_j)=f|_{W_j}(Z\cap W_j)\subseteq Y
	\]
	is closed, whence so is the subset $f(Z)=\bigcup_{j=1}^nf(Z\cap W_j)$.
\end{proof}

\begin{prp}
	Let $X$ and $Y$ be two topological spaces, and
	let $f\colon\fromto{X}{Y}$ be a map.
	Then the following are equivalent.
	\begin{itemize}
		\item The map $f$ is closed.
		\item For any $y\in Y$ and any open set $U\supseteq X_y$,
			there is an open neighbourhood $V$ of $y$ such that $f^{-1}(V)\subseteq U$.
	\end{itemize}
\end{prp}

\begin{proof}
	Assume that $f$ is closed, and
	assume that $y\in Y$ and $U\supseteq X_y$ open.
	Then set $V\coloneq Y \smallsetminus f(X \smallsetminus U)$;
	clearly $y\in V$, and since $f$ is closed, $V$ is open.
	Now an element of $f^{-1}(V)$ does not lie in $f^{-1}(f(X \smallsetminus U))$, whence
	we deduce that it does not lie in $X \smallsetminus U$, whence it lies $U$, as desired.

	Conversely, suppose the second condition is satisfied, and
	suppose $Z\subseteq X$ a closed subset.
	By the second condition, for any point $y\notin f(Z)$, there is an open neighbourhood $V$ of $y$ such that $f^{-1}(V)\subseteq X \smallsetminus Z$, whence we deduce that $Y \smallsetminus f(Z)$ is open.
\end{proof}

\begin{prp}\label{prp:compactness}
	The following are equivalent for a topological space $X$.
	\begin{itemize}
		\item Any open cover $\UU$ of $X$ contains a finite \defn{subcover} --
			i.e., a finite subset $\UU_0\subseteq\UU$ that covers $X$ as well.
		\item Let $T$ be a topological space, and let $t\in T$ be a point.
			Then for any open subset $U\subseteq T\times X$ that contains $\{t\}\times X$, there is an open neighbourhood $V$ of $t$ such that $V\times X\subseteq U$.
		\item For any topological space $T$, the projection map $\pr_1\colon\fromto{T\times X}{T}$ is closed.
		\item Let $\ZZ\subseteq\PP(X)$ be a collection of closed subsets of $X$ such that $\cap\ZZ=\varnothing$;
			then there exists a finite subset $\ZZ_0\subseteq\ZZ$ such that the intersection $\cap\ZZ_0=\varnothing$.
	\end{itemize}
\end{prp}

\begin{proof}
	Let us assume the first condition and prove the second.
	So let $T$ be a topological space and $t\in T$, and
	let $U$ an open set of $T\times X$ containing $\{t\}\times X$.
	Now let $\UU$ be the collection of those open subsets $W\subseteq X$ such that for some open neighbourhood $E$ of $t$, one has $E\times W\subseteq U$.
	The collection $\UU$ covers%
	\sidenote{Since $U$ is open in the product topology, for every $x\in X$, the point $(t,x)$ lies in an open neighbourhood of the form $E\times W$ such that $E\times W\subseteq U$}
	$X$.
	The first condition now ensures that there is a finite subcover $\UU_0\subseteq\UU$.
	For each $W\in\UU_0$, select an open neighbourhood $E_W$ of $t$ such that $E_W\times W\subseteq U$.
	Now set
	\[
		V\coloneq\bigcap_{W\in\UU_0}E_W;
	\]
	so that $V$ is an open neighbourhood of $t$, and $V\times W\subseteq U$ for any $W\in\UU_0$. Hence
	\[
		V\times X=\bigcup_{W\in\UU_0}V\times W\subseteq U.
	\]
	This verifies the second condition.

	The equivalence of the second and third conditions is the content of the previous proposition.

	Let us assume the third condition and prove the fourth.
	So let $\ZZ\subseteq\PP(X)$ be a collection of closed subsets of $X$, and
	assume that for any finite subset $\ZZ_0\subseteq\ZZ$, the intersection $\cap\ZZ_0$ is nonempty.
	Now let $T$ be the set $X\sqcup\{\infty\}$, which we topologise with the coarsest topology such that%
	\sidenote{Please note that this is weird! The topology we're putting on $T$ really has nothing to do with the topology we have on $X$;
		it's really all about $\ZZ$.
	We're using the class $\ZZ$ of \emph{closed} subsets of $X$ to describe some \emph{open} sets of $T$.}
	\begin{itemize}
		\item \emph{any} subset $S\subseteq X$ is open in $T$, and
		\item for any $Z\in\ZZ$, the set $Z\sqcup\{\infty\}$ is open in $T$.
	\end{itemize}
	Any open neighbourhood of $\infty$ contains a finite intersection of elements of $\ZZ$.
	Since all such intersections are nonempty, any open neighbourhood of $\infty$ contains at least one point of $X$.
	Hence $\infty$ is not open or, equivalently, $X\subseteq T$ is dense.
	Now let
	\[
		\Delta \coloneq \{(x,x) : x\in X\} \subseteq T\times X\comma
	\]
	and consider the closed set $\pr_1(\overline{\Delta})\subseteq T$.
	Clearly $X=\pr_1(\Delta)\subseteq\pr_1(\overline{\Delta})$, so $\pr_1(\overline{\Delta})=T$.
	In other words, $\overline{\Delta}$ contains a point $(\infty, x)$ for some $x\in X$.
	Thus for every $Z\in\ZZ$ and every open neighbourhood $U$ of $x$, one has $((Z\sqcup\{\infty\})\times U)\cap\Delta\neq\varnothing$, and so%
	\sidenote{since a point $(u,u)$ of that intersection has the property that $u\in Z\cap U$}
	$Z\cap U\neq\varnothing$.
	Thus $x$ is close to every $Z\in\ZZ$, whence $x$ lies in every $Z\in\ZZ$, whence $x\in\cap\ZZ$.

	Finally, let's assume the fourth condition and prove the first.
	Let $\UU\subseteq\PP(X)$ be an open cover of $X$.
	Now let $\ZZ$ be the set of complements of elements%
	\sidenote{Hence $Z\in\ZZ$ if and only if $X \smallsetminus Z\in\UU$.}
	of $\UU$.
	Clearly $\cap\ZZ=X-\cup\UU=\varnothing$, so by the fourth condition, there exists a finite subset $\ZZ_0\subseteq\ZZ$ such that $\cap\ZZ_0=\varnothing$.
	Now let $\UU_0$ be the set of complements of elements%
	\sidenote{Hence $U\in\UU_0$ if and only if $X \smallsetminus U\in\ZZ_0$.}
	of $\ZZ_0$.
	Now $\cup\UU_0=X \smallsetminus \cap\ZZ_0=X$, whence $\UU_0\subseteq\UU$ is a finite subcover.
\end{proof}

\begin{dfn}
	A topological space $X$ is \defn{compact}%
	\sidenote{Bourbaki use the term \defn{quasicompact} for the notion we have introduced here, and reserve the term \defn{compact} for topological spaces that are also \defn{Hausdorff}.
	The Bourbaki convention is typical in French mathematical literature and in algebraic geometry literature.}
	if and only if it satisfies any of the equivalent conditions of \ref{prp:compactness}.
	In this case, $X$ is called a \defn{compactum}.
\end{dfn}

\begin{nul}
	Note that compactness is a \defn{topologically invariant} notion;
	that is, if $X$ and $Y$ are homeomorphic topological spaces, then $X$ is compact if and only if $Y$ is compact.%
	\sidenote{Please observe that this cannot be said for notions like openness or closedness;
	those depend upon the \emph{ambient topological space}.}
\end{nul}

\begin{exm}
	Any finite topological space is compact, and the indiscrete topology on any set is compact.
\end{exm}

\begin{exm}
	The discrete compacta are all finite.
\end{exm}

\begin{prp}
	Any closed interval $[a,b]$ is compact.
\end{prp}

\begin{proof}
	Suppose $\UU$ an open covering of $[a,b]$;
	then let
	\begin{equation*}
		c=\sup\left\{x\in[a,b] : [a,x]\text{ is contained in the union of finitely many elements of }\UU\right\} \period
	\end{equation*}
	Now suppose, to generate a contradiction, that $c<b$;
	then let $U\in\UU$ be an element of the open cover containing $c$.
	Then for some $\varepsilon>0$, one has $\left]c-\varepsilon,c+\varepsilon\right[\subset U$.
	By definition of $c$, one has $[a,c-\varepsilon/2]\subseteq\cup\UU_0$ for a finite subset $\UU_0\subseteq\UU$;
	hence the union of the elements of $\UU_0\cup\{U\}\subseteq\UU$ contains $[a,c+\varepsilon/2]$, yielding the desired contradiction.
	Thus $c=b$.

	Now choose an element $V\in\UU$ than contains $b$;
	for some $\varepsilon>0$, one has $\left]b-\varepsilon,b\right]\subset U$.
	Now since $c=b$, there exists a finite subset $\VV_0\subseteq\UU$ such that $\left[a,b-\varepsilon/2\right[\subseteq\cup\VV_0$, and so $\VV_0\cup\{V\}\subseteq\UU$ is a finite cover of $[a,b]$.
\end{proof}

\begin{exm}
	No Euclidean space $\RR^n$ is compact;%
	\sidenote{Please note that this example, combined with the previous one, implies that $\RR$ is not homeomorphic to any closed interval.
	There are other ways to prove this, but this one really cuts through the treacle!}
	indeed, the open balls $B^n(x,\varepsilon)$ form an open cover, but there is no finite subcover.
	To see this, consider any finite cover of $\RR^n$ by balls $B^n(x_i,\varepsilon_i)$ for $i=1,2,\dots,m$, and suppose $x$ a point of one of the balls.
	Then one may choose a real number $r$ so that $B^n(x,r)\supset\bigcup_{i=1}^mB^n(x_i,\varepsilon_i)$.
\end{exm}

\begin{exm}
	Any set with the cofinite topology is compact.
\end{exm}

\begin{nul}
	If $(X,\tau)$ is a compactum, and if $\tau'$ is a topology on $X$ that is coarser than $\tau$, then $(X,\tau')$ is also a compactum.
\end{nul}

\begin{lem}
	Let $X$ and $Y$ be topological spaces, and
	let $f\colon\fromto{X}{Y}$ be a continuous surjection.
	If $X$ is compact, so is $Y$.
\end{lem}

\begin{proof}
	Let $T$ be a topological space.
	If $Z\subseteq T\times Y$ is closed, then $f^{-1}(Z)\subseteq X$ is closed, so%
	\sidenote{The displayed equation is an \emph{abuse of notation}: on the left side is the direct image under the projection $\pr_1\colon\fromto{T\times Y}{T}$, and
	on the right is the direct image under the projection $\pr_1\colon\fromto{T\times X}{T}$.}
	\[
		\pr_1(Z)=\pr_1(f^{-1}(Z))
	\]
	is closed as well.
\end{proof}

\begin{lem}
	A closed subspace of a compactum is compact.
\end{lem}

\begin{proof}
	Let $X$ be a compactum, and suppose $Z\subseteq X$ closed.
	For any topological space $T$, the projection $\pr_1\colon\fromto{T\times X}{T}$ is closed, and
	the inclusion $T\times Z \inclusion T\times X$ is closed, so the composite $\fromto{T\times Z}{T}$, which is the projection, is closed.
\end{proof}

\begin{prp}%
	\sidenote{This is the easy case of Tychonoff's Theorem, to which we shall return.}
	Let $\{X_i\}_{i\in\{1,2,\dots,n\}}$ be a family of compacta.
	Then the product $\prod_{i=1}^nX_i$ is compact.
\end{prp}

\begin{proof}
	Suppose $T$ a topological space.
	Then the projections $\fromto{T\times X_1}{T}$, $\fromto{T\times X_1\times X_2}{T\times X_1}$, \ldots, and $\fromto{T\times X_1\times X_2\times\cdots\times X_n}{T\times X_1\times\cdots\times X_{n-1}}$ are all closed, and so their composite
	\[
		\fromto{T\times X_1\times X_2\times\cdots\times X_n}{T}
	\]
	is closed as well.
\end{proof}

\begin{thm}\label{thm:heineborel}
	The following are equivalent for a subspace $A\subset\RR^n$.
	\begin{itemize}
		\item $A$ is compact.
		\item {[Bolzano--Weierstra{\ss}]} Every sequence of points in $A$ has a convergent subsequence.
		\item {[Heine--Borel]} $A$ is closed and bounded.
	\end{itemize}
\end{thm}
\begin{proof}
	Let us show that the first of these conditions implies the second.
	Suppose $A$ compact, and suppose $(x_i)_{i\geq 0}$ a sequence of points in $A$.
	If there were no convergent subsequence of $(x_i)_{i\geq 0}$, then the set $\{x_i\}_{i\geq 0}$ would be a closed subset of $A$, hence compact, and
	there would be a sequence $\varepsilon_i$ of positive real numbers such that the balls $B^n(x_i,\varepsilon_i)$ would be disjoint.
	But then $\{B^n(x_i,\varepsilon_i)\}_{i\geq 0}$ would be an open cover of $\{x_i\}_{i\geq 0}$ with no finite subcover.

	Let us show that the second property implies the third.
	Suppose $A$ has the property that every sequence has a convergent subsequence.
	Then $A$ must be bounded, since otherwise there exists a sequence $(x_i)_{i\geq 0}$ of points in $A$ such that $\|x_i\|\to\infty$.
	It must also be closed, since if $x\in\overline{A}-A$, one can construct a sequence of points of $A$ converging to $x$.

	Finally, let us show that the third condition implies the first.
	Suppose $A$ is closed and bounded.
	Since $A$ is bounded, it is contained in a box $[-a,a]^n$ for some $a\geq 0$.
	Since every closed subspace of a compact space is compact, it is enough to show that $[-a,a]^n$ is compact.
	Since a finite product of compacta is compact, it is enough to show that any closed interval is compact, and we've already shown this.
\end{proof}

\begin{cor}
	Suppose $X$ a compactum, and suppose $f\colon\fromto{X}{\RR}$ a continuous function.
	Then $f$ attains both a maximum and a minimum value.
\end{cor}

