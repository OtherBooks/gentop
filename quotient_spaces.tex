%!TEX root = rootfile.tex 
% chktex-file 3
% chktex-file 8
% chktex-file 12
% chktex-file 24
% chktex-file 42
	
\begin{nul}
	Let $X$ be a set.
	Recall that an \emph{equivalence relation} $\sim$ on $X$ is a relation that satisfies the following conditions:
	\begin{itemize}
		\item For every $x \in X$, one has $x \sim x$.
		\item For every $x,y \in X$, one has $x \sim y $ if and only if $y \sim x $.
		\item For every $x,y,z \in X$, if $x \sim y$ and $y \sim z$, then $x \sim z$.
	\end{itemize}
	If $ x \in X$, one can form the \emph{equivalence class} $[x] \coloneq \{ y \in X : y \sim x \}$.
	Now the set $X/\!\sim \colon \{ [x] : x \in X \}$ of equivalence classes relative to $\sim$ is called the \emph{quotient} of $X$ relative to $\sim$.
	The assignment $ x \mapsto [x]$ is a map $ q \colon X \to X/\!\sim$, which is called the \emph{quotient map};
	The map $q$ is a surjection, and it enjoys the following pleasant property:
	for every set $Y$ and every map $f \colon X \to Y$ with the property that for every $x, y \in X$, if $x \sim y$, then $f(x) = f(y)$, there exists a unique map $f' \colon X/\!\sim \to Y$
	such that $f' \circ q = f $.

	Every surjection is secretly a quotient map: if $ p \colon X \to T $ is a surjection, then we may define an equivalence relation $ \sim $ on $X$ in which $x \sim y$ if and only if $p(x) = p(y)$.
	We may call this the equivalence relation \emph{induced by} $p$.
	The unique map $p' \colon X/\!\sim \to T$ through which $p$ factors is then a bijection.

	Finally, if $R$ is \emph{any} relation on $X$, then it \emph{generates} an equivalence relation.
	That is, recall that $\sim$ is \emph{really} a subset%
	\sidenote{consisting of those pairs $(x,y)$ such that $x R y$}
	of $ X \times X $.
	If we look at the collection of all subsets $Q \in \PP(X \times X)$ that are equivalence relations that contain $R$, then the intersection of all these $Q$ is itself an equivalence relation that contains $R$.
	This intersection is the \emph{equivalence relation generated by} $R$.
\end{nul}

\begin{dfn}
	Let $X$ be a topological space, let $Y$ be a set, and let $p \colon X \to Y $ be a surjection.
	The final topology with respect to $p$ is the \defn{quotient topology} on $Y$;
	this is the finest topology on $Y$ such that the map $p$ is continuous.
	We call $p$ the \defn{quotient map}.
	We also say that $Y$ is the \emph{quotient space} of $X$ under the equivalence relation induced by $p$.
\end{dfn}

The ability to form quotients of topological spaces is really the most powerful motivation for defining abstract topological spaces in the first place.

\begin{exm}
	Consider the real line $\RR$, and consider the equivalence relation $\sim$ in which $x \sim y $ if and only if $y-x$ is an integer.
	Now consider the map $t \mapsto \exp(2\pi t)$; this is a continuous map $e \colon \RR \to S^1$.
	By the discussion above, there exists a unique map $ f \colon \RR/\!\sim \to S^1$ such that if $q \colon \RR \to \RR/\!\sim $ is the quotient map, then $f \circ q = e$.
	This map is continuous with respect to the quotient topology on $\RR/\sim$.
	In fact, $f$ is a homeomorphism.

	Similarly, consider the closed interval $[0,1]$, and consider the equivalence relation $\sim$ generated by the requirement that $0\sim 1$.
	Then the inclusion map $[0,1] \inclusion \RR$ descends to a continuous map $[0,1]/\sim \to \RR/\!\sim$, and this map is a homeomorphism as well.
	This justifies our intuition that the circle is obtained from the interval by identifying the endpoints.
\end{exm}

\begin{exm}
	Let $X \subseteq \RR^n$ be a subspace, and let $k \in \NN$ be a natural number.
	Then we have the subspace $X^k \subseteq \RR^{nk}$.
	Furthermore, we can define the subsspace
	\[
		Y \coloneq \{(x_1,\dots,x_k) \in X^k : (\forall i \neq j)(x_i \neq x_j) \} \subseteq X^k \period
	\]
	Now conisder the equivalence relation $\sim$ on $Y$ in which $(x_1, \dots, x_k) \sim (y_1,\dots, y_k)$ if and only if there exists a permutation%
	\sidenote{We write $\Sigma_k$ for the set of permutations of the set $\{1,\dots,k\}$.}
	$\sigma \in \Sigma_k$ such that for each $i$, one has $y_i = x_{\sigma(i)}$.
	Now we can think of the points of the quotient space $C_k(X) \coloneq Y/\!\sim$ as \emph{unordered} collections of $k$ points on $X$.
	This space $C_k(X)$ is called the \defn{configuration space} of $k$ marked points on $X$.

	The space $C_0(X)$ has exactly one point.

	The space $C_1(X)$ is $X$ itself.

	The space $C_2(X)$ is much more interesting.
	For example, $C_2(\RR^2) \cong \RR^3 \times S^1$.
	The space $C_2(S^1)$ is the \emph{M\"obius band}.

	As $k$ goes higher, these spaces become even more involved.

	The spaces $C_k(X)$ are extremely important in physics, since we may think of them as the space of possible states of $k$ noninteracting particles on $X$.
\end{exm}

\begin{exm}
	Let $V$ be a finite dimensional real vector space.
	Since $V$ is a finite dimensional, it is isomorphic to $\RR^n$.
	Let us transport the topology from $\RR^n$ to $V$ along such an isomorphism.%
	\sidenote{That is, if $\phi \colon V \to \RR^n$ is a linear isomorphism, then $U \subseteq V $ is open if and only if $\phi(V) \subseteq \RR^n$ is open.
		As an exercise, you can show that the topology is the same, irrespective of which isomorphism $\phi$ you select.
	(Hint: linear isomorphisms from $\RR^n$ to itself are always homeomorphisms!)}
	Thus $V$ is homeomorphic to $\RR^n$.

	For every $k \in \NN^{\ast}$, the set $V^k$ of $k$-tuples $(v_1, \dots, v_k)$ of vectors in $V$ is also a vector space, and thus also a topological space that is homeomorphic to $\RR^{nk}$.
	Now we let $L_{\RR}(k,V) \subseteq V^k$ be the set of $k$-tuples $(v_1, \dots, v_k)$ of vectors in $V$ that are linearly independent,
	so that they span a $k$-dimensional subspace of $V$.

	Now we define an equivalence relation $\sim$ on $L_{\RR}(k, V)$ by the rule that $(v_1, \dots, v_k) \sim (w_1, \dots, w_k)$ if and only if their spans are equal (as subspaces of $V$).
	The quotient space $G_{\RR}(k,V) \coloneq L_{\RR}(k,V)/\!\sim$ is called the \defn{Grassmannian} of $k$-dimensional subspaces of $V$.
	The points of $G_{\RR}(k,V)$ are $k$-dimensional real linear subspaces of $V$.
	One often writes $G_{\RR}(k,n)$ as a shorthand for $G_{\RR}(k, \RR^n)$.

	In the particular case in which $k=1$, the space $G_{\RR}(1,n)$ is the space of lines through the origin of $\RR^n$;
	this is also called the (real) \defn{projective space} $\PP^{n-1}_{\RR}$.

	We can do this same thing with $\CC$ in place%
	\sidenote{That is, we can look at a finite dimensional complex vector space V with the topology transported from $\CC^n \cong \RR^{2n}$;
		we can consider the subspace $L(k, V)$ of $\CC$-linearly independent $k$-tuples of $V^k$;
	and we can contemplate the quotient $G(k, V)$ of $k$-dimensional complex subspaces of $V$.}
	of $\RR$.

	In particular, $G(k,n)$ is the space of \emph{complex} lines through the origin in $\CC^n$;
	this is called the (complex) \defn{projective space} $\PP^{n-1}$.
\end{exm}

\begin{nul}
	Let $X$ and $Y$ be topological spaces, and let $p \colon X \to Y $ be a continuous surjection.
	Then $p$ need not be a quotient map.
	For example, the map $e \colon \left]0,1\right] \to S^1$ given by $t \mapsto \exp(2\pi t)$ is not a quotient map: even though $e^{-1}(e(\left]1/2,1/2\right])) = \left]1/2,1/2\right]$ is open, the subset $e(\left]1/2,1/2\right]) \subseteq S^1$ is not.
	If $p$ is an \emph{open map} -- that is, if it carries open sets to open sets -- then it is a quotient map.
\end{nul}

\begin{nul}
	Let $\{X_a\}_{a\in A}$ be an indexed family of sets.
	Then the \defn{coproduct of sets} is by definition the union
	\[
		\coprod_{a\in A}X_a\coloneq\bigcup_{a\in A}X_a\times\{a\} \period
	\]
	It is engineered precisely so that a map out of the coproduct is determined by each of its components.
	More precisely, the coproduct is equipped with inclusion maps
	\[
		\iota_b\colon\fromto{X_b}{\coprod_{a\in A}X_a} \comma
	\]
	one for every $b\in A$, given by $\iota_b(x)=(x,b)$.
	For any set $Y$, and for any collection $\{ f_a \colon X_a \to Y\}_{a\in A}$ of maps, there exists a unique map
	\[
		f \colon \coprod_{a \in A} X_a \to Y
	\]
	such that for each $a\in A$, one has $f \circ \iota_a = f_a$.
	Thus composition with the inclusions induces a bijection
	\[
		\prod_{a\in A}-\circ \iota_{a} \colon\fromto{\Map\left(\coprod_{a\in A}X_a,Y\right)}{\prod_{a\in A}\Map(X_a,Y)} \period
	\]
\end{nul}


\begin{dfn}
	Let $X$ be a topological space, and let $A\subseteq X$ a subspace.
	Consider the map $q\colon\fromto{X}{(X-A)\sqcup\{X-A\}}$ defined by
	\[
		q(x)\coloneq \begin{cases} x&\text{if }x\notin A;\\
			X-A&\text{if }x\in A.
		\end{cases}
	\]
	We define $X/A$ as the set $(X-A)\sqcup\{X-A\}$ equipped with the final topology with respect to $q$.
\end{dfn}

\begin{exm}
	Consider the subspace
	\[D^{n}\coloneq\{x\in\RR^{n} : \|x\|\leq 1\}\subset\RR^{n}\]
	and the subspace
	\[S^{n-1}\coloneq\{x\in D^{n} : \|x\|=1\}\subset D^{n} \period\]
	Then the space $D^n/S^{n-1}$ is homeomorphic to $(\RR^n)^+$, which is in turn homeomorphic to $S^n$.
\end{exm}

\begin{dfn}
	Let $\{(X_a,\tau_a)\}_{a\in A}$ be an indexed family of topological spaces.
	Then the \defn{coproduct topology} $\coprod_{a\in A}\tau_a$ is the final topology with respect to the set $\{\iota_a\}_{\alpha\in A}$.
\end{dfn}

\begin{nul}
	Let $S=\coprod_{a\in A}S_a\subseteq\coprod_{a\in A}X_a$ (so that $S_a\subseteq X_a$).
	If $\tau$ is the coproduct topology, then
	\[
		\tau(S)=\coprod_{a\in A}\tau_a(S_a) \period
	\]
\end{nul}

\begin{exm}
	If $X$ is topological space, then $X_+$ denotes the coproduct $X\sqcup\{X\}$ with the coproduct topology.
\end{exm}

\begin{exm}
	Let $A = (A,\delta)$ be a discrete topological space.
	Suppose $f\colon\fromto{X}{A}$ a continuous map.
	Then we have a bijection $G \colon \coprod_{a\in A}f^{-1}\{a\} \to X$.
	Let us show that this is a homeomorphism, where the right hand side is given the coproduct topology (and the fibers $f^{-1}\{a\} \subseteq X$ are given the subspace topology).
	By construction, the map $G$ is continuous.
	It remains to show that $G^{-1}$ is continuous;
	let $U \subseteq \coprod_{a\in A}f^{-1}\{a\}$ be an open subset.
	Then $U \cap f^{-1}\{a\}$ is open in $f^{-1}\{a\}$, and so since $U$ is the union of these intersections, it suffices to prove that any open subset of $f^{-1}\{a\}$ is open in $X$.
	Indeed, this is true, since $f^{-1}\{a\}$ is itself open.

	Conversely, if $\{X_a\}_{a\in A}$ is an indexed family of topological spaces, then define the map
	\[
		g\colon\fromto{\coprod_{a\in A}X_a}{A}
	\]
	such that $g(x)=a$ if and only if $x\in X_a$.
	The map $g$ is continuous.

	In other words, a decomposition of a toploogical space into a coproduct of topological spaces is equivalent to a continuous map into a discrete topological space.
\end{exm}


Much of the story of topology is the story of ``gluing topological spaces together'' to get new ones.
We are now ready to say exactly what this means.

\begin{exm}
	Let $U$, $V$, and $X$ be three topological spaces, and let $f\colon\fromto{X}{U}$ and $g\colon\fromto{X}{V}$ be two continuous maps.
	Consider the equivalence relation $\sim$ on $U\sqcup V$ generated by declaring that:
	for every element $x\in X$, $f(x) \sim g(x)$.
	Write $U \cup^X V$ for the quotient $(U\sqcup V)/\sim$ with the quotient topology.

	Consider the subset
	\[D^{n}\coloneq\{x\in\RR^{n} : \|x\|\leq 1\}\subset\RR^{n}\]
	and the subset
	\[S^{n-1}\coloneq\{x\in D^{n} : \|x\|=1\}\subset D^{n}.\]
	Here's a standard way to build a new space from an old one. Suppose $X$ a topological space, and suppose $f\colon\fromto{S^{n-1}}{X}$ a continuous map. Then we can form the \defn{cell attachment for attaching map $f$}
	\[
		X\cup^{S^{n-1}}D^n.
	\]
	A \defn{cell complex} is a topological space $Y$ that is built via cell attachments from discrete spaces; that is, a cell complex is a topological space $Y$ along with a sequence of subspaces
	\[
		Y_0\subseteq Y_1\subseteq \cdots\subseteq Y
	\]
	such that:
	\begin{itemize}
		\item $Y=\bigcup_{n\geq 0}Y_n$;
		\item $Y_0$ is discrete;
		\item for any $n\geq 1$, there exist an indexed family $\left\{f_{\alpha}\colon\fromto{S^{n-1}}{Y_{n-1}}\right\}_{\alpha\in A_n}$ of attaching maps such that
	\[
		Y_n=Y_{n-1}\cup^{\coprod_{\alpha\in A_n}S^{n-1}}\coprod_{\alpha\in A_n}D^{n}.
	\]
	\end{itemize}
\end{exm}


