%!TEX root = rootfile.tex 
% chktex-file 3
% chktex-file 8
% chktex-file 12
% chktex-file 24
% chktex-file 42

\begin{dfn}
	Let $X \subseteq \RR^n$ be a subspace.
	A subset $ S \subseteq X$ that is both open and closed (in $X$) is said to be \defn{clopen}.%
	\sidenote{I know, this is silly.}
	We shall say that $X$ is \defn{connected} if there are \emph{exactly two} clopen subsets $S \subseteq X$.
\end{dfn}

\begin{nul}
	If $X \subseteq \RR^n$ is a nonempty subspace, then there are always \emph{at least two} clospen subsets of $X$:
	the empty set $\varnothing$, and $X$ itself.
	Thus if $X$ is nonempty, then it is connected if and only if the only nonempty clopen subset is $X$ itself.
\end{nul}

\begin{exm}
	The empty set $\varnothing$, however, is not connected: it has only one subset, itself, which is clopen.
\end{exm}

\begin{exm}
	The space $\RR$ itself is connected.
	Let's prove this.
	There are at least two clopen subsets: $\varnothing$ and $\RR$ itself.
	Now we have to prove that there are no more.
	So let $S \subseteq \RR$ be a clopen subset.
	If $S \neq \varnothing$, then there exists a point $x \in S$.
	Since $S$ is open, there exists $\varepsilon >0$ such that $\left]x-\varepsilon, x+\varepsilon\right[ \subseteq S$.
	Now let us consider the set
	\[
		E \coloneq \left\{ \varepsilon >0 : \left]x-\varepsilon, x+\varepsilon\right[  \subseteq S \right\} \subseteq \RR \period
	\]
	The set $E$ is unbounded if and only if $S = \RR$, so assume that $E$ in bounded.
	We aim to generate a contradiction.
	Since $E$ is bounded, it admits a supremum $\varepsilon_0$.
	In particular, for every $\varepsilon < \varepsilon_0$, one has $\left]x-\varepsilon, x+\varepsilon\right[  \subseteq S$.
	Note that $\left]x-\varepsilon_0, x+\varepsilon_0\right[  = \bigcup_{\varepsilon < \varepsilon_0} \left]x-\varepsilon, x+\varepsilon\right[ $, so it follows that $\left]x-\varepsilon_0, x+\varepsilon_0\right[ \subseteq S$ as well.
	Thus $\varepsilon_0 \in E$.

	Now consider the point $x+\varepsilon_0$.
	This is close to $S$, because the closure of $S$ contains the closure of $\left]x-\varepsilon_0, x+\varepsilon_0\right[$, which is $[x-\varepsilon_0,x+\varepsilon_0]$. 
	Since $S$ is closed, it follows that $x +\varepsilon_0 \in S$.
	The same analysis shows that $x-\varepsilon_0 \in S$.

	Again since $S$ is open, we may choose a $\delta>0$ such that $\left]x+\varepsilon_0-\delta, x+\varepsilon_0+\delta\right[ \subset S$ \emph{and} $\left]x-\varepsilon_0-\delta, x-\varepsilon_0+\delta\right[ \subset S$.
	But now we have that $\left]x-\varepsilon_0-\delta, x+\varepsilon_0+\delta\right[ \subset S$, so $\varepsilon_0+\delta \in E$, which contradicts the maximality of $\varepsilon_0$.

	This contradiction shows that $E$ is unbounded, and so the only nonempty clopen subset of $\RR$ is $\RR$ itself.
\end{exm}

\begin{exm}
	The subspace
	\[
		Y \coloneq \{(x,y) \in \RR^2 : |x|=1 \}
	\]
	we introduced above is not connected.
	Indeed, in addition to $\varnothing $ and $Y$ itself, the subset
	\[
		Y_+ \coloneq \{(x,y) \in \RR^2 : x=1 \}
	\]
	is clopen.
	To see this, let us show that both $Y_+$ and its complement
	\[
		Y_- \coloneq Y \smallsetminus Y_+ = \{ (x,y) \in \RR^2 : x = -1 \}
	\]
	are open.
	The key observation here is that if $ u \in Y_+$ and $v \in Y_-$, then $d(u,v) \geq 2$.
	Consequently, if $u \in Y_+$, then $B^2(u,1) \cap Y \subset Y_+$, and, similarly, if $v \in Y_-$, then $B^2(v,1) \cap Y \subset Y_-$.
\end{exm}

\begin{prp}%
\label{prp:intersecting_unions_of_connecteds_are_connected}
	Let $\{X_a\}_{a\in A} $ be a nonempty family of connected subspaces $X_a \subseteq \RR^n$.
	Assume that for any $a,b \in A$, one has $X_a\cap X_b \neq \varnothing$.
	Then the union $x \coloneq \bigcup_{a\in A}X_a $ is connected as well.
\end{prp}

\begin{proof}
	Since $X_a \cap X_b$ is nonempty, the union $X$ is nonempty too.
	Hence it suffices to show that if $ V \subseteq X $ is a nonempty clopen, then $ V = X$.
	
	Note that for any $ a\in A$, the intersection $ V \cap X_a $ is clopen in $X_a$.
	For each $ a\in A$, the subspace $X_a$ is connected, so $V \cap X_a$ is either $\varnothing$ or $X_a$.
	In other words, for each $a \in A$, either $V$ is disjoint from $X_a$ or else $X_a \subseteq V$.
	Since $V\neq\varnothing$, there is at least one $ a_0 \in A $ such that $ X_{a_0} \subseteq V$.
	
	But now for any other $a\in A$, the nonempty intersection $X_a \cap X_{a_0}$ is contained in $ X_a \cap V $;
	thus $X_a \subseteq V $ as well.
	We thus conclude that $X \subseteq V$.
\end{proof}

We are now in a position to classify all the connected subspaces of $\RR$.

\begin{exm}
	Let $X \subseteq\RR$ be a nonempty subset.
	We'll say that $X$ is \emph{an interval} if and only if, for every $a, b \in X$ and every $x \in \RR $ such that $a \leq x \leq b$, we have $ x\in X$.
	Assume that $X$ is an interval.
	If $X$ is bounded above, then there exists a supremum $b \in \RR$.
	If $X$ is bounded below, then there exists an infimum $a \in \RR$.
	Now, depending upon whether $a$ and $b$ exist, there are three options for an interval $X$:
	\begin{itemize}
		\item a interval of finite length such as $[a,b]$, $\left]a,b\right]$, $\left[a,b\right[$, or $\left]a,b\right[$;
		\item a ray $\left]a,+\infty\right[$, $\left[a,+\infty\right[$, $\left]-\infty, b\right[$, or $\left]-\infty, b\right]$; or
		\item the line $\RR$ itself.
	\end{itemize}
	
	Here now is our claim:
	a subspace $X \subseteq \RR$ is connected if and only if it is an interval -- hence if and only if it is of one of the three forms above.
	To prove this, we must prove two things:
	\begin{itemize}
		\item first, that any interval is connected, and
		\item second, that any subspace that is not an interval is not connected.
	\end{itemize}

	To prove the first statement, assume first that $X$ is a closed interval $[a,b]$.
	Assume that $V \subseteq [a,b]$ is a clopen subset.
	Suppose that $x \in [a,b]$ is a point such that $x \notin V$;
	we aim to show that $V$ is empty.
	
	If there are points $ s\in V$ such that $ s < x$, let
	\[
		s_0 \coloneq \sup \{s\in V : s< x\} \period
	\]
	Thus $s_0 \leq x$.
	Observe that $s_0$ is close\sidenote{\Cref{exm:sups_and_infs_are_close}} to $V$.
	Since $V$ is closed, it follows that $s_0 \in V$.
	Since $x \notin V$, it follows that $s_0<x$.
	Since $V$ is open, there is a $\varepsilon >0$ such that $\left]s_0-\varepsilon,s_0+\varepsilon\right[ \cap [a,b]$ is contained in $V$.
	In particular, there exist points $s \in V$ such that $s_0 < s < x$, which contradicts the definition of $s_0$.
	Consequently, there are no points $s\in V$ such that $s<x$.

	On the other hand, if there are points $s \in V$ such that $s>x$, then let
	\[
		s_1 \coloneq \inf \{s\in V : s>x\} \period
	\]
	We can replay the same argument as above to generate a contradiction.
	Since $V$ is closed, $s_1 \in V$.
	Since $V$ is open, there are points $ s\in V$ such that $ x<s<s_1$, which is a contradiction.
	It therefore follows that $V=\varnothing$, as desired.

	So far, we have shown that a closed interval $[a,b]$ is always connected.
	Now for a general interval $X$, note that if $x\in X$, then we may write
	\[
		X = \bigcup_{\substack{a,b\in X, \\ a\leq x\leq b}} [a,b] \period
	\]
	This is a union of connected subspaces of $\RR$ such that any two intersect (because they all contain $x$).
	Hence $X$ is connected as well, thanks to the previous proposition.

	Now let us prove the converse.
	Assume that $X \subset \RR$ is \emph{not} an interval.
	We aim to prove that $X$ is not connected;
	it will suffice to construct a nonempty proper clopen $ V \subseteq X$.
	For this, choose real numbers $a<x<b$ such that $a,b \in X$ but $x \notin X$.
	Now consider the subset
	\[
		V = \left[x,+\infty\right[ \cap X = \left]x,+\infty\right[ \cap X \period
	\]
	The first expression proves that $V$ is closed; the second proves that $V$ is open.
	Since $b \in V$ but $a\notin V$, it follows that $V$ is nonempty and proper.
\end{exm}

Here's our first truly \emph{topological} theorem.
We will think of this as a generalization of the intermediate value theorem.
\begin{thm}%
\label{thm:IVT}
	Let $ f \colon X \to Y $ be a continuous surjection.
	If $X$ is connected, so is $Y$.
\end{thm}

\begin{proof}
	Assume that $X$ is connected.
	In particular, $X$ is nonempty, hence so is $Y$.
	
	Let $ V \subseteq Y $ be a nonempty clopen.
	Then $f^{-1}(V) \subseteq X$ is clopen since $f $ is continuous, and it is nonempty since $f$ is a surjection.
	Since $X$ is connected, it follows that $f^{-1}(V) = X$ itself.
	This implies that the image $f(X)$ is contained in $V$.
	Since $f$ is a surjection, it follows that $V=Y$.
	Thus $Y$ has exactly two clopen subsets: $\varnothing$ and $Y$.
\end{proof}

\begin{exm}
	Let $X \subseteq \RR^n$ be a connected subspace, and let $f \colon X \to \RR$ be a continuous map.
	Then the image $f(X)$ is an interval.
	This is our friend, the intermediate value theorem.

	Consider the following assertion: there are two \defn{antipodal}%
	\sidenote{Two points are antipodal if and only if they are in exactly opposite directions from the center of the Earth.}
	points on the Earth's equator that, at ground level, have exactly the same temperature.%
	\sidenote{For the purposes of this discussion, we assume that temperature is a continuous function of position.}
	In other words, there are two coordinates of the form
	\[
		0^{\circ}\text{S,} \; x^{\circ}\text{E} \andeq 0^{\circ}\text{N,} \; (180-x)^{\circ}\text{W}
	\]
	where at ground level the temperature is exactly the same.
	
	Here's the proof.
	The Earth's equator is homeomorphic to the circle $S^1$, which is a connected subspace of $\RR^2$.
	Temperature at the equator is thus a continuous function $T \colon S^1 \to \RR$.
	Now define a related continuous map $ g \colon S^1 \to \RR$ by the formula
	\[
		g(x) = T(x)-T(-x) \period
	\]
	Now since $x$ and $-x$ are antipodal points, our claim will be proved if we can show that there is a point $x\in S^1 $ such that $g(x)=0$.
	If $g(1,0)=0$, then we are done.
	Otherwise, our formula ensures that $g(-1,0)=g(1,0)$, so $g(1,0)$ and $g(-1,0)$ have different signs.%
	\sidenote{That is, one is positive and one is negative.}
	But since the image $g(S^1) \subseteq\RR$ is an interval, it follows that $ 0 \in g(S^1)$.
	Hence there exists $ x \in S^1$ such that $g(x)=0$.
\end{exm}

\begin{exm}
	Since $\RR \smallsetminus\{0\} $ and the subspace
	\[
		Y = \{(x,y) \in \RR^2 : |x|=1\}
	\]
	are homeomorphic, it follows that $\RR\smallsetminus \{0\}$ is not connected.

	Since $\RR$ is connected, it follows that it is not homeomorphic to $\RR \smallsetminus \{0\}$.
\end{exm}

\begin{exm}
	A good way to confirm that a nonempty subspace $X \subseteq \RR^n$ is connected is to confirm that you can \enquote{walk} from every point to every other point without leaving $X$.
	For any two points $x,y \in X$, a \defn{path} from $x$ to $y$ in $X$ is a continuous map
	\[
		\gamma \colon [0,1] \to X
	\]
	such that $\gamma(0)=x$ and $\gamma(1) = y $.
	If $X$ has the property that for every pair of points from $x$ to $y$, there exists a path from $x$ to $y$ in $X$, then $X$ is connected.%
	\sidenote{The converse is, annoyingly, false.
	This is the difference between \emph{connectedness} and what is called \defn{path connectedness}.}
	Why? Well, choose $x\in X$.
	For every path $\gamma$ from $x$ to another point in $X$, consider the image $\gamma([0,1])$;
	this is connected, and our assumption on $X$ ensures that $X$ is the union of all these images, all of which contain $x$.
	Hence $X$ is connected.
\end{exm}

\begin{exm}
	The line $\RR$ is not homeomorphic to $S^1$.
	Indeed, suppose it were; choose a homeomorphism $f \colon \RR \to S^1$.
	This will restrict to a homeomorphism $\RR \smallsetminus \{0\} \to S^1\smallsetminus\{f(0)\}$.
	However, we claim that $S^1 \smallsetminus \{f(0)\} $ is connected.
	In fact, we'll do better: we shall prove that $S^1 \smallsetminus \{f(0)\}$ is homeomorphic to $\RR$.

	To make our formulas simpler, let's use complex numbers.%
	\sidenote{The field of complex numbers $\CC$ is really nothing more than $\RR^2$, equipped with its complex multiplication rule $(a,b)\cdot (c,d) = (ac-bd, ad+bc)$.}
	We have:
	\[
		S^1 = \{z \in \CC : |z| = 1 \}.
	\]
	Let $w = f(0)$.
	We may first construct a homeomorphism $S^1\smallsetminus\{w\} \to S^1 \smallsetminus \{1\} $ by the rule $ z \mapsto z/w $.
	Since the inverse map is $ z \mapsto wz$, which is also continuous, we are done.

	With this done, we now construct a homeomorphism $\left]0,1\right[ \to S^1 \smallsetminus \{1\}$ by the rule $t \mapsto \exp(2\pi i t) $.
	This map is a continuous bijection, but what about its inverse $r$?
	To make it easy to deduce the continuity of $r$ from well-known bits of analysis, we write it as:%
	\sidenote{Please note, however, that this formula will not work to define a continuous map from $S^1 \to \RR$.}
	\[
		r(x,y) = \frac{1}{2}-\frac{1}{\pi}\arctan\left(\frac{y}{1-x}\right) \period
	\]
	It now follows that $S^1\smallsetminus\{1\}$ is homeomorphic to an open interval and thus to $\RR$ itself.
\end{exm}

